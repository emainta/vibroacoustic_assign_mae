\documentclass[a4paper,12pt,oneside]{article}

% Imported packages
\usepackage[english]{babel}
\usepackage[T1]{fontenc}
\usepackage[utf8]{inputenc}
\usepackage{float}
\usepackage{graphicx}
\usepackage{amssymb}
\usepackage{amsmath}
\usepackage{bm}
\usepackage{listings}
\usepackage{comment}
\usepackage{geometry}
\usepackage{wrapfig}
\usepackage{subfigure}

% Margin dimensions settings
\geometry{a4paper,top=2cm,bottom=2cm,left=2cm,right=2cm,%
heightrounded,bindingoffset=5mm}

% Enumeration settings
\renewcommand\thesubsection{\thesection.\alph{subsection}}

% Code visualization settings
\lstset{basicstyle=\small\ttfamily}

% Code design settings
\lstset{language=Matlab}

% Included images path
\graphicspath{{Images/}}

% Document information
\title{Fundamentals of Vibration Analysis and Vibroacoustics \\
Module 1 - Fundamentals of Vibration Analysis \\
Assignment 3 - Modal parameter identification}
\author{Bombaci Nicola 10677942 \\
Fantin Jacopo 10591775 \\
Intagliata Emanuele 10544878}
\date{May 2020}


\begin{document}

\maketitle

\vspace{100pt}

In this problem, we're given a small set of measurements we have to extract information from. The virtual experiment consists in a hammer hitting impulsively a structure in one point, and measuring the displacements in four different positions, one of which is where the force is applied (the first one). The initial data are wherefore the time axis $ t $ used to represent the sampled measurements and the force, the impulse-like input force $ F(t) $, and the four measurements:

\[
	t ~ \text{,} ~ F(t) ~ \text{,} ~ %
		x_1(t) ~ \text{,} ~ x_2(t) ~ \text{,} ~ x_3(t) ~ \text{,} ~ x_4(t)
\]

\section{Experimental FRFs}
\label{sec:experimental_frfs}

The Frequency Response Functions linking the displacement of the four points to the external force can be retrieved computing the ratio between the Discrete Fourier Transform (DFT) of the former and of the latter. We used the Matlab function \lstinline! fft() ! to find the DFT of the function input, here denoted with $ \mathcal{F[\cdot]} $, through an optimized Fast Fourier Transform algorithm.

%\begin{lstlisting}
%	H_exp = [fft(x1)./fft(F);
%		 fft(x2)./fft(F);
%		 fft(x3)./fft(F);
%		 fft(x4)./fft(F)];
%\end{lstlisting}

\[
	H^\textup{exp}_k(f) = \frac{\mathcal{F}[x_k(t)]}{\mathcal{F}[F(t)]} ~ \text{,} ~ %
		k = 1,2,3,4
\]

Note that the first element $ H^\textup{exp}_1(f) $ represents the co-located FRF. The corresponding plots are represented in magnitude and phase as follows:

\begin{figure}[H]
	\centering
	\includegraphics[scale=0.4]{experimental_FRFs_modulus}
\end{figure}

\begin{figure}[H]
	\centering
	\includegraphics[scale=0.4]{experimental_FRFs_phase}
\end{figure}


\section{Simplified methods for model identification}

To estimate the system parameters, we first reached for simplified methods. The assumption we are working under are, as usual, low damping (order of magnitude for $ \xi $: $ 10{-2} $), and well-distincted peaks in amplitude spectrum. As long as the damped natural frequencies are concerned, we found the relative maxima in each FRF setting one order of magnitude lower than the absolute maximum as threshold, and then found the frequencies corresponding to those values. The resulting matrix of damped natural frequencies is:

\[
	\mathbf{f_d} =	\begin{bmatrix}
										f_{\textup{d}_1}^{(1)}	& f_{\textup{d}_1}^{(2)} & %
											f_{\textup{d}_1}^{(3)}	& f_{\textup{d}_1}^{(4)} \\
										f_{\textup{d}_2}^{(1)}	& f_{\textup{d}_2}^{(2)} & %
											f_{\textup{d}_2}^{(3)}	& f_{\textup{d}_2}^{(4)} \\
										f_{\textup{d}_3}^{(1)}	& f_{\textup{d}_3}^{(2)} & %
											f_{\textup{d}_3}^{(3)}	& f_{\textup{d}_3}^{(4)} \\
										f_{\textup{d}_4}^{(1)}	& f_{\textup{d}_4}^{(2)} & %
											f_{\textup{d}_4}^{(3)}	& f_{\textup{d}_4}^{(4)} \\
									\end{bmatrix} = \begin{bmatrix}
																		1.0500	& 1.9667	& 2.6667	& 3.0833 \\
																		1.0500	& 1.9833	& 2.6667	& 3.0667 \\
																		1.0500	& 1.9833	& 2.6667	& 3.0667 \\
																		1.0500	& 1.9833	& 2.6667	& 3.0667
																	\end{bmatrix}
\]

where every row corresponds to one of the four different measurements, and each column to a vibration mode. The values are affected by small fluctuations as for the 2\textsuperscript{nd} and 4\textsuperscript{th} frequency, but only for the first measurement. We can therefore reckon this is just due to numerical errors.

As a second step, we searched for the adimensional damping ratios, applying the half-power bandwidth method. This exploits the relationship between the width of a resonating mode bell-like peak and the damping ratio, which are dirctly proportional. Each ratio for each measurement is computed thanks to the formula

\[
	\xi_i = \frac{(f^{(i)}_2)^2 - (f^{(i)}_1)^2}{4 \, (f^{(i)}_\textup{d})^2} %
		~ \text{,} ~ i = 1,2,3,4
\]

$ i $ being the mode number. The resulting matrix of damping ratios is:

\[
	\bm{\xi} =	\begin{bmatrix}
								\xi_1^{(1)}	& \xi_1^{(2)} & \xi_1^{(3)}	& \xi_1^{(4)} \\
								\xi_2^{(1)}	& \xi_2^{(2)} & \xi_2^{(3)}	& \xi_2^{(4)} \\
								\xi_3^{(1)}	& \xi_3^{(2)} & \xi_3^{(3)}	& \xi_3^{(4)} \\
								\xi_4^{(1)}	& \xi_4^{(2)} & \xi_4^{(3)}	& \xi_4^{(4)} \\
							\end{bmatrix} = \begin{bmatrix}
																0.0159	& 0.0128	& 0.0062	& 0.0081 \\
																0.0159	& 0.0084	& 0.0062	& 0.0054 \\
																0.0159	& 0.0128	& 0.0094	& 0.0054 \\
																0.0159	& 0.0126	& 0.0062	& 0.0081
															\end{bmatrix}
\]

organized like the damped natural frequency matrix. This time, the values vary a little more than before: this is due to non-negligible variations of the peaks width relative to the same resonance in distinct measurements, especially for mode 2 in measurement 2 and mode 3 in measurement 3. As for the 4\textsuperscript{th} mode, conformity between measurement 1 and 4 and between measurement 2 and 3 may be observed. Values for the 1\textsuperscript{st} mode are the most accurate both for the frequency value and for the damping ratio, in accordance to the fact the more the resonances are %low in frequency, the less they "pack up", the easier it is to make an estimation. In other words, they correspond to clearer and higher peaks.
clear and their peaks are sharp for each measurement, the easier it is to make an estimation, because values will tend to be the same for all of the measurements. In fact, the first mode corresponds to a very clear peak for each of them.

Finally, the mode shape of i-th mode is estimated equalling the experimental FRF for measurement $ k = 1,2,3,4 $, computed in the previous section and evaluated at the i-th resonance frequency, $ i = 1,2,3,4 $, to the approximated FRF $ H^{\textup{approx}^{(i)}}_k $ connecting displacement of point $ k $ and the force, evaluated at the resonant frequencies. The mode shape element relative to the force application position has been neglected, being it 

\[ \begin{split}
	& H^{\textup{exp}}_k(\omega_{\textup{d}_k}^{(i)}) \approx %
		H^{\textup{approx}^{(i)}}_k(\omega_{\textup{d}_k}^{(i)}) = %
		\frac{\phi^{(i)}_k} %
		{-m^{(i)}_k \, \omega^2 + j \, \omega \, c^{(i)}_k + k^{(i)}_k} %
		\Big|_{\omega = \omega_{\textup{d}_k}^{(i)}} = %
		\frac{\phi^{(i)}_k}{j \, \omega_{\textup{d}_k}^{(i)} \, c^{(i)}_k} = %
		-j \, \frac{\phi^{(i)}_k}{\omega_{\textup{d}_k}^{(i)} \, c^{(i)}_k} \\
	& \Rightarrow ~ %
		j \, \frac{\phi^{(i)}_k}{\omega_{\textup{d}_k}^{(i)} \, c^{(i)}_k} = %
		- H^{\textup{exp}}_k(\omega_{\textup{d}_k}^{(i)}) %
		~ \Rightarrow ~ \frac{\phi^{(i)}_k}{\omega_{\textup{d}_k}^{(i)} \, c^{(i)}_k} = %
		\Im\{- H^{\textup{exp}}_k(\omega_{\textup{d}_k}^{(i)})\} = %
		- \Im\{H^{\textup{exp}}_k(\omega_{\textup{d}_k}^{(i)})\} \\
	& \Rightarrow ~ \phi^{(i)}_k = %
		- \Im\{H^{\textup{exp}}_k(\omega_{\textup{d}_k}^{(i)})\} %
		\, \omega_{\textup{d}_k}^{(i)} \, c^{(i)}_k ~ \text{,} ~ i,k = 1,2,3,4
\end{split} \]

where the contributions of other modes other than the i-th mode has been neglected in the computation of the i-th mode shape. The damping coefficients $ c^{(i)}_k $ have been derived from the damping ratios following the definition of damping ratio:

\[ \begin{split}
	\xi^{(i)}_k = \frac{c^{(i)}_k}{2 \, m^{(i)}_k \, \omega_{\textup{d}_k}^{(i)}} %
		~ \Rightarrow ~ c^{(i)}_k = 2 \, \xi^{(i)}_k \, m^{(i)}_k \, %
		\omega_{\textup{d}_k}^{(i)}
\end{split} \]

and setting the elements of the modal mass matrix $ m^{(i)}_k $ to 1.
The modal matrix of all the elements $ \phi^{(i)}_k $ is reported here:

\[
	\bm{\phi} =	\begin{bmatrix}
								\phi_1^{(1)}	& \phi_1^{(2)} & \phi_1^{(3)}	& \phi_1^{(4)} \\
								\phi_2^{(1)}	& \phi_2^{(2)} & \phi_2^{(3)}	& \phi_2^{(4)} \\
								\phi_3^{(1)}	& \phi_3^{(2)} & \phi_3^{(3)}	& \phi_3^{(4)} \\
								\phi_4^{(1)}	& \phi_4^{(2)} & \phi_4^{(3)}	& \phi_4^{(4)} \\
							\end{bmatrix} = \begin{bmatrix}
																0.1050	& 0.3438	& 0.2130	& 0.0683 \\
																0.1642	& 0.1065	& -0.1725	& -0.1071 \\
																0.1519	& -0.2704	& -0.1083	& 0.1195 \\
																0.0735	& -0.2841	& 0.2303	& -0.1465
															\end{bmatrix}
\]

Normalizing the elements of each measure taking the first one as reference the result is

\[
	\bm{\phi} =	\begin{bmatrix}
								1				& 1				& 1				& 1 \\
								1.5632	& 0.3098	& -0.8098	& -1.5669 \\
								1.4459	& -0.7867 & -0.5087	& 1.7482 \\
								0.7000	& -0.8264 & 1.0811	& -2.1442 \\
							\end{bmatrix}
\]

\vspace{10pt}

These values are in agreement with the magnitude and phase plots of \ref{sec:experimental_frfs}: the modulus of $ \phi^{(i)}_k $ is approximatedly the relative value in the amplitude plots, while its sign follows the sign of the phase in the phase plots. The biggest divergence between the absolute value of the elements in the modal matrix and the plotted values is about measurement 3, 3\textsuperscript{rd} mode, which should have a relative magnitude of $ \frac{1}{3} $ with respect to the 1\textsuperscript{st} measurement, but the corresponding element in $ \bm{\phi} $ indicates $ \frac{1}{2} $. This is certainly due to the approximated values of the modal matrix, obtained by means of a simplified method. In the next section, we go on with the analysis using a more accurate procedure, which is a modal parameter identification model.

% Da qui parte ema
\section{Modal parameter identification}
Modal identification uses the modal approach as analytical support, meaning that
it reconstructs the analytical transfer function $H_k( \omega) $ of the system considering it as if it were made up of many one d.o.f. systems, since the various d.o.f. are
defined by the modal variables. $H_k ( \omega) $ represents the harmonic transfer function for an input at the generic point k (in our system di output position is fixed) of the N d.o.f. system considered. So this transfer function will have modal parameters as
unknown quantities:

\[
	H_k ( \omega ) = H_k (\omega, \omega_{d_k}^{(i)} , m_k ^{(i)}, k_k ^{(i)},  \xi_k ^{(i)}, 	    	\mathbf{ \phi }^{(i)} )	
\]

being:

\begin{itemize}
\item $ \omega_{d_k} $ the natural frequencies;
\item $ m_k ^{(i)} $ the generalised masses;
\item $ k_k ^{(i)} $ the generalised stifness;
\item $ \xi_k ^{(i)} $ the modal damping;
\item $	\mathbf{ \phi }^{(i)} $ the modes of vibration.
\end{itemize}

With this approach the dynamic response of the structure subjected to the known excitation is measured at several k points; this dynamic response, expressed in terms of experimental
transfer functions $H_k ^{EXP} ( \omega) $ , is then compared to the analytical response $H_k ^{NUM} ( \omega) $ defined beforehand by minimising the difference between the analytical values and the experimental ones. It is thus possible to determine the set of modal parameters needed to characterise the static and dynamic behaviour of the system being analysed.

From the modal approach we know that, as far as a sufficient number of modes N are considered, the following expression is always valid:

\[
H_{jk} (\omega) = \sum_{i=1}^{N} \dfrac{\phi_{j}^{(i)} \phi_{k}^{(i)}}
{-\omega^2 m_i +j \omega c_i + k_i}
\]
considering a system forced at location j, whose response is measured at location k.


For well distinguished peaks and lightly damped structures, the analytical Frequency Response Function $H_k ^ {NUM} ( \omega) $ of the system can be approximated around a certain $\omega_i$ as:

\[	
H_{jk}^{NUM} (\omega) = \dfrac{A_{j}^{(i)} + jB_{j}^{(i)}}
{-\omega^2 m_i +j \omega c_i + k_i}
+ (C_{j}^{(i)} + jD_{j}^{(i)})
+ \dfrac{E_{j}^{(i)} + jF_{j}^{(i)}}{\omega^2}
\]

In this form:

\begin{itemize}
\item the first term represent the contribution of the \textbf{resonating mode} in our frequency range of interest;
\item the second term represent the contribution of the modes at \textbf{higher} frequency than $\omega_i$; this contribution is approximately constant because our frequency range of interest falls in the \textbf{quasi-static zone} of this modes;
\item the third term represent the contribution of the modes at \textbf{lower} frequency than $\omega_i$; this contribution has approximately a $\dfrac{1}{\omega^2}$ because our frequency range of interest falls in the \textbf{seismographic region} of this modes.
\end{itemize}

INSERIRE FIGURA DELLA SLIDE 21 - LAB 2 (PARTE 2)

The frequency ranges of interest are partitions of the frequency axis. Given the N well distinguished peaks we will have N ranges of interest. The bounds of this ranges are the local minima between one peak and the following one. 

INSERIRE IMMAGINE ESEMPLIFICATIVA matlab DEI MINIMI E MASSIMI INDIVIDUATI
 
\subsection{Vibration modes identification procedure}
For a given set of experimental FRFs $H_k ^ {EXP} ( \omega) $ , obtained for a fixed excitation location j (that for this reason will be omitted) and different measurement locations k (4 points), a least squares minimization procedure can be implemented for the estimation of the modal parameters.

The error function to be minimized is:

\[
\varepsilon = \sum_{s=s_inf}^{s_sup} 
\Re{ \lbrace
H_k ^ {EXP} ( \omega_s) - H_k ^ {NUM} ( \omega_s)  
\rbrace }^2 +
\Im{ \lbrace
H_k ^ {EXP} ( \omega_s) - H_k ^ {NUM} ( \omega_s)  
\rbrace}^2
\]

Since the error function depends non-linearly from the unknown parameters, an iterative
minimization procedure is used. This is implemented in \textsc{Matlab} throught the function 
\texttt{fminsearch}. According to the documentation, \texttt{fminsearch} is a nonlinear programming solver. Itearches for the minimum of a problem specified by
$\min\limits_{x} f(x)$. 
Its syntax is: \texttt{x = fminsearch(fun,x0,options)}.

\texttt{fun} is the function to minimize, specified as a function handle or function name. \texttt{fun} is a function that accepts a vector or array x and returns a real scalar f (the objective function evaluated at x).

An initial guess vector \texttt{x0} (in our code is \texttt{xpar0}) is required, consisting of a preliminary estimate of:
\begin{enumerate}
\item $\omega_i$ which is found from the maximum peak in the considered frequency range. We already discussed the procedure in the \ref{sec:simplified_methods} section. It has been evaluated from each FRF and then averaged.
\item $\xi_i$ which is found through a simplified method (in our case the half power bandwidth). We already discussed the procedure in the \ref{sec:simplified_methods} section. It has been evaluated from each FRF and then averaged.
\item  $A_{j}^{(i)}$ which is found considering each FRF at resonance and assuming real
valued mode shapes (valid in resonance condition):

\[
\phi^{(i)}_k = %
		- \Im\{H^{\textup{exp}}_k(\omega_{\textup{d}_k}^{(i)})\} %
		\, \omega_{\textup{d}_k}^{(i)} \, c^{(i)}_k
		= A_{j}^{(i)}
\]

We already discussed the procedure in the \ref{sec:simplified_methods} section.
It follows that $B_{j}^{(i)} = 0 $ .
\item $C_{j}^{(i)} , D_{j}^{(i)}, E_{j}^{(i)}, 	F_{j}^{(i)}$ are set to zero under the assumption of sufficiently distinguished peaks.
\end{enumerate}

The non-linear minimization procedure elaborates separately the FRFs, leading to an estimate of the modal parameters $(\omega_i, \xi_i, A_{j}^{(i)})$ for every frequency range (i.e. for every peak) and for every measurement. Then the parameters are averaged. 
A further improvement of the algorithm could be to perform the minimization simultaneously on the whole set of FRFs, leading to a more precise estimate of the modal parameters $(\omega_i, \xi_i, A_{j}^{(i)})$ for every peak. In this case the information giver by the correlation of the FRFs is exploited.
The quality of the estimates can be visually assessed comparing in a plot the
identified FRFs $H_k ^ {NUM}$ with the experimental ones $H_k ^ {EXP}$.

INSERIRE PLOT matlab DELLE FRF RICOSTRUITE



\end{document}
