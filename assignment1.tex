\documentclass[a4paper,12pt,oneside]{article}

% Imported packages
\usepackage[english]{babel}
\usepackage[T1]{fontenc}
\usepackage[utf8]{inputenc}
\usepackage{float}
\usepackage{graphicx}
\usepackage{amssymb}
\usepackage{amsmath}

% Enumeration settings
\renewcommand\thesubsection{\thesection.\alph{subsection}}

% Included images path
\graphicspath{{Images/}}

% Document information
\title{Fundamentals of Vibration Analysis and Vibroacoustics \\
Module 1 - Fundamentals of Vibration Analysis \\
Assignment 1 - One-degree-of-freedom systems}
\author{Bombaci Nicola \\
Fantin Jacopo 10591775 \\
Intagliata Emanuele}
\date{April 2020}


\begin{document}

\maketitle

\section{Equation of motion}

\subsection{Equation derivation}

\subsubsection*{Step 1: number of degrees of freedom identification}

We can verify the system has one degree of freedom since

\[ \begin{split}
n_b \cdot 3 \text{DOF} & - \\
2 \text{DOF} & - \text{(hinge)} \\
2 \text{DOF}  & - \text{(2 rollers)} \\
1 \text{DOF} & = \text{(string)} \\
1 \text{DOF}
\end{split} \]

We chose to solve the problem directly using Lagrange equation, so to have one equation only, as the system has one degree of freedom.

\subsubsection*{Step 2: energy terms definition}

\[ E_c = \frac{1}{2} \, J_1 \, \omega_1^2 + \frac{1}{2} \, M_2 \, v_2^2 \]

\[
V_e = \frac{1}{2} \, k_1 \, \Delta l_1^2 + \frac{1}{2} \, k_2 \, \Delta l_2^2 \, \text{;} %
\quad V_g = M_2 \, g \, h_2
\]
\[
\Rightarrow V = V_e + V_g = \frac{1}{2} \, k_1 \, \Delta l_1^2 + \frac{1}{2} \, k_2 \, \Delta l_2^2 + %
M_2 \, g \, h_2
\]

\[ D = \frac{1}{2} \, c_1 \, \dot{\Delta l_1}^2 + \frac{1}{2} \, c_2 \, \dot{\Delta l_2}^2 \]

Because the assignment's requests define external forces to compute the system's forced motion later on, we're assuming a vertical force $ F(t) $, directed upward, applied on $ M_2 $, so that we'll find a positive Lagrangian component.

\[ \delta W = F(t) \, \delta y_2 \]

\subsubsection*{Step 3: phisical variables as functions of independent ones}

The independent variable $ \theta $ is chosen to be the one variable we need to describe the motion.

\begin{flalign}
  \omega_1 = \dot{\theta} && \nonumber
\end{flalign}
\begin{flalign}
  v_2 = \dot{y_2} = \omega_1 \, R_2 = \dot{\theta} \, R_2 && \nonumber
\end{flalign}
\begin{flalign}
  \dot{\Delta l_1} = \dot{\theta} \, R_2 \quad \text{(Rivals theorem)} \nonumber \\ %
  \Rightarrow \Delta l_1 = \theta \, R_2 && \nonumber
\end{flalign}
\begin{flalign}
  \dot{\Delta l_2} = - \dot{\theta} \, R_1 \quad \text{(Rivals theorem)} \nonumber \\ %
  \Rightarrow \Delta l_2 = - \theta \, R_1 && \nonumber
\end{flalign}
\begin{flalign}
  h_2 = y_2 = \theta \, R_2 %
  \quad \text{(gravitational potential level $ = 0 $ at equilibrium point)} && \nonumber
\end{flalign}
\begin{flalign}
  \delta y_2 = \delta \theta \, R_2 && \nonumber
\end{flalign}

\subsubsection*{Step 4: resulting equation}

\[
\delta_W = Q_\theta \, \delta \theta = F(t) \, \delta y_2 = %
F(t) \, \delta \theta \, R_2 \Rightarrow Q_\theta = F(t) \, R_2
\]

\begin{align}
  \frac{\partial}{\partial t} %
  \Bigl(\frac{\partial E_c}{\partial \dot{\theta}}\Bigr) - %
  \frac{\partial E_c}{\partial \theta} + %
  \frac{\partial V}{\theta} + %
  \frac{\partial D}{\partial \dot{\theta}} & = Q_\theta \notag \\
  (J_1 + M_2 \, R_2^2) \, \ddot{\theta} + (c_1 \, R_2^2 + c_2 \, R_1^2) \, \dot{\theta} + (k_1 \, R_2^2 + k_2 \, R_1^2) \, \theta & = F(t) \, R_2 - M_2 \, g \, R_2 \notag
\end{align}

\subsection{Adimensional damping ratio}



\subsection{Natural and damped frequency}



\section{Free motion of the system}



\subsection{Generic initial conditions}



\subsection{Halved adimensional damping ratio}



\subsection{Raised adimensional damping ratio}



\section{Forced motion of the system}



\subsection{Frequency Response Function}



\subsubsection*{Case 1: generic initial conditions}



\subsubsection*{Case 2: halved adimensional damping ratio}



\subsubsection*{Case 3: raised adimensional damping ratio}



\subsection{Temporal evolution of complete response of the system}



\subsubsection*{Case 1: f = 0.15Hz}



\subsubsection*{Case 2: f = 4.5Hz}



\subsection{Forced response of the system (steady-state)}




\end{document}
